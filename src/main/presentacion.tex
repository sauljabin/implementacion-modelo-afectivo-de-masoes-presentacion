\documentclass{beamer}
\usetheme{UCLA}

\usepackage[utf8]{inputenc}
\usepackage[T1]{fontenc}

%% Fuente
\usepackage{helvet}

\title{Implementación de un Modelo Afectivo para la Arquitectura Multiagente para Sistemas Auto-Organizados y Emergentes (MASOES)}
\subtitle{Maestría en Ciencias de la Computación, Mención Inteligencia Artificial}
\author{Ing. Saúl Piña}
\date{Octubre 25, 2017}
\institute{\url{sauljabin@gmail.com}}

\begin{document}

\begin{frame}[plain,t]
\titlepage
\end{frame}

\begin{frame}
\frametitle{Agenda}
\begin{itemize}
\item Introducción
\item Objetivo General
\item MASOES
\item Modelo Afectivo de MASOES
\item Propuesta
\item Casos de Estudio
\item Demostración
\item Conclusión
\item Preguntas
\end{itemize}
\end{frame}

\section{Introducción}

\begin{frame}
\frametitle{Introducción}

\end{frame}

\begin{frame}
\frametitle{Objetivo General}
\huge
Implementar el modelo afectivo de MASOES en un sistema multiagente
\end{frame}

\section{MASOES}

\begin{frame}
\frametitle{MASOES}
La una arquitectura multiagente para sistemas emergentes y auto-organizados llamada
MASOES (Multiagent Architecture for Self-Organizing and Emergent Systems, en inglés),
es una herramienta para el diseño no
formal de sistemas, que produzcan un estado auto-organizado el cual emerja de
las interacciones locales entre los agentes y de los cambios que se dan en el
entorno. En esta arquitectura, cada agente puede cambiar su comportamiento
dinámicamente, guiado por su estado emocional, para satisfacer dinámicamente los
objetivos del sistema a través de la auto-organización de sus actividades.
\end{frame}

\begin{frame}
\frametitle{MASOES}
\framesubtitle{Componentes de MASOES a Nivel Individual}
\centering
\includegraphics[width=7cm]{ilustraciones/componentes-masoes-individual}
\end{frame}

\begin{frame}
\frametitle{MASOES}
\framesubtitle{Componente Conductual}
\centering
\includegraphics[width=5cm]{ilustraciones/componente-conductual}
\end{frame}

\begin{frame}
\frametitle{MASOES}
\framesubtitle{Modelo Afectivo de MASOES}
El modelo afectivo (o emocional) de MASOES considera un conjunto de
emociones positivas y negativas generadas desde un nivel individual o colectivo,
para de esta manera promover un comportamiento individual (Reactivo, Cognitivo)
o colectivo (Imitativo) en los agentes y así, aumentar su grado de satisfacción
y por consecuencia, el nivel de auto-organización y emergencia general en el
sistema.
\end{frame}

\begin{frame}
\frametitle{MASOES}
\framesubtitle{Modelo Afectivo de MASOES}
\begin{columns}
\column{0.5\textwidth}
Este modelo afectivo está representado por un espacio bidimensional,
donde el eje $x$ representa el nivel de \textbf{Activación} del agente en el
intervalo $[-1, 1]$, y el eje $y$ representa el nivel de \textbf{Satisfacción},
también en el intervalo $[-1, 1]$.
\column{0.5\textwidth}
\includegraphics[width=5cm]{ilustraciones/modelo-afectivo}
\end{columns}
\end{frame}

\begin{frame}
\frametitle{MASOES}
\framesubtitle{Comportamientos Según el Estado Emocional del Agente}
\begin{table}[!ht]
\centering
\scriptsize
\begin{tabular}{ccc}
\hline
\bfseries Emoción & \bfseries Tipo de Emoción & \bfseries Comportamiento \\
\hline\hline
Felicidad & Positivo & Imitación \\
Alegría & Positivo & Imitación \\
Compasión & Positivo & Imitación \\
Admiración & Positivo & Imitación \\
Tristeza & Ligeramente Negativo & Cognitivo \\
Rechazo & Ligeramente Negativo & Cognitivo \\
Depresión & Altamente Negativo & Reactivo \\
Ira & Altamente Negativo & Reactivo \\
\hline
\end{tabular}
\end{table}
\end{frame}

\begin{frame}
\frametitle{MASOES}
\framesubtitle{Reglas de Priorización de Comportamientos}
\begin{table}[!ht]
\centering
\scriptsize
\begin{tabular}{ll}
\hline
\textbf{Regla 1:} & Si el \textit{Estado Emocional} es \textit{Positivo} \\
& entonces priorizar \textit{Comportamiento Imitativo} \\
\textbf{Regla 2:} & Sino Si el \textit{Estado Emocional} es \textit{Ligeramente Negativo} \\
& entonces priorizar \textit{Comportamiento Cognitivo} \\
\textbf{Regla 3:} & Sino Si el \textit{Estado Emocional} es \textit{Altamente Negativo} \\
& entonces priorizar \textit{Comportamiento Reactivo} \\
\hline
\end{tabular}
\end{table}
\end{frame}

\section{Propuesta}

\begin{frame}
\frametitle{Propuesta}
\framesubtitle{Aspectos Arquitecturales}
\begin{columns}
\column{0.5\textwidth}
\textbf{JADE}\\
\textbf{Java Agent
DEvelopment}, uno de los marcos de trabajo con paradigma de POA (\textbf{Programación Orientada a Agentes})
más populares, implementado en el lenguaje de programación Java
\column{0.5\textwidth}
\textbf{FIPA}\\
\textbf{Foundation for Intelligent Physical Agents},
las cuales representan una colección de normas que tienen como objetivo promover la interoperabilidad
de agentes heterogéneos y los servicios que pueden representar
\end{columns}
\end{frame}

\begin{frame}
\frametitle{Propuesta}
\framesubtitle{Aspectos Arquitecturales}
\centering
\includegraphics[width=8cm]{ilustraciones/arquitectura}
\end{frame}

\begin{frame}
\frametitle{Propuesta}
\framesubtitle{Comunicación Entre Agentes}
\centering
\includegraphics[width=8cm]{ilustraciones/comunicacion-entre-hosts}
\end{frame}

\begin{frame}
\frametitle{Aspectos Propuestos a Nivel Individual}
\framesubtitle{Propuesta de Una Ontología Para MASOES}
\begin{columns}
\column{0.5\textwidth}
\centering
\tiny
\includegraphics[width=5cm]{ilustraciones/ontologia-masoes-estado}
\\
Acción Consultar Estado del Agente
\column{0.5\textwidth}
\centering
\tiny
\includegraphics[width=5cm]{ilustraciones/ontologia-masoes-estimulo}
\\
Acción Evaluar Estímulo
\end{columns}
\end{frame}

\begin{frame}
\frametitle{Aspectos Propuestos a Nivel Individual}
\framesubtitle{Protocolo de Comunicación Para Peticiones FIPA}
\centering
\includegraphics[width=4cm]{ilustraciones/flujo-fipa-protocolo-request}
\end{frame}

\begin{frame}
\frametitle{Aspectos Propuestos a Nivel Individual}
\framesubtitle{Diseño del Agente Emocional}
\centering
\includegraphics[width=7cm]{ilustraciones/diseno-nivel-individual}
\end{frame}

\begin{frame}
\frametitle{Aspectos Propuestos a Nivel Individual}
\framesubtitle{Diseño del Componente Conductual}
\centering
\includegraphics[width=9cm]{ilustraciones/diseno-componente-conductual}
\end{frame}

\begin{frame}
\frametitle{Componente Conductual}
\framesubtitle{Base de Conocimiento Conductual}
\centering
Conocimiento Relacionado al Agente en la BCC
\begin{table}[!ht]
\centering
\tiny
\begin{tabular}{ll}
\hline
\scriptsize \bfseries Cláusula & \scriptsize \bfseries Descripción \\
\hline
\hline
yo(agente). & Definición del agente actual. \\\hline
otro(A) :- not yo(A). & Definición de otro agente. \\ & A representa el nombre del agente \\
\hline
\end{tabular}
\end{table}
\end{frame}

\begin{frame}
\frametitle{Componente Conductual}
\framesubtitle{Base de Conocimiento Conductual}
\centering
Conocimiento Relacionado a las Emociones en la BCC
\begin{table}[!ht]
\centering
\tiny
\begin{tabular}{ll}
\hline
\scriptsize \bfseries Cláusula & \scriptsize \bfseries Descripción \\
\hline
\hline
tipo\_emocion(admiracion, positiva). & La admiración es positiva \\\hline
tipo\_emocion(compasion, positiva). & La compasión es positiva \\\hline
tipo\_emocion(felicidad, positiva). & La felicidad es positiva \\\hline
tipo\_emocion(alegria, positiva). & La alegría es positiva \\\hline
tipo\_emocion(rechazo, ligeramente\_negativa). & El rechazo es ligeramente negativa \\\hline
tipo\_emocion(tristeza, ligeramente\_negativa). & La tristeza es ligeramente negativa \\\hline
tipo\_emocion(ira, altamente\_negativa). & La ira es altamente negativa \\\hline
tipo\_emocion(depresion, altamente\_negativa). & La depresión es altamente negativa \\
\hline
\end{tabular}
\end{table}
\end{frame}

\begin{frame}
\frametitle{Componente Conductual}
\framesubtitle{Base de Conocimiento Conductual}
\centering
Conocimiento Relacionado a las Comportamientos en la BCC
\begin{table}[!ht]
\centering
\tiny
\begin{tabular}{ll}
\hline
\scriptsize \bfseries Cláusula & \scriptsize \bfseries Descripción \\
\hline
\hline
prioridad\_comportamiento(E, imitativo) & El comportamiento es imitativo si la \\
:- tipo\_emocion(E, positiva). & emoción E es positiva \\\hline
prioridad\_comportamiento(E, cognitivo) & El comportamiento es cognitivo si la \\
:- tipo\_emocion(E, ligeramente\_negativa). & emoción E es ligeramente negativa \\\hline
prioridad\_comportamiento(E, reactivo) &  El comportamiento es reactivo si la  \\
:- tipo\_emocion(E, altamente\_negativa). & emoción E es altamente negativa \\
\hline
\end{tabular}
\end{table}
\end{frame}

\begin{frame}
\frametitle{Componente Conductual}
\framesubtitle{Base de Conocimiento Conductual}
\centering
Ejemplos de Conocimiento de Estímulos en la BCC
\begin{table}[!ht]
\centering
\tiny
\begin{tabular}{ll}
\hline
\scriptsize \bfseries Cláusula & \scriptsize \bfseries Descripción \\
\hline
\hline
estimulo(A, saludar, 0.01, 0.01) :- otro(A). & Para el agente A el estímulo saludar aumentará \\ & la activación en 0.01 y la satisfacción en 0.01 \\ & si el estímulo fue enviado por otro agente \\ \hline
estimulo(A, despertar, 0.02, -0.1) :- yo(A). & Para el agente A el estímulo despertar aumentará \\ & la activación en 0.02 y disminuirá la satisfacción \\ & en 0.1 si el evento fue generado por el mismo agente \\
\hline
\end{tabular}
\end{table}
\end{frame}

\begin{frame}
\frametitle{Componente Conductual}
\framesubtitle{Configurador Emocional}
\centering
Modelo Afectivo

\includegraphics[width=6cm]{ilustraciones/modelo-afectivo-propuesto}
\end{frame}

\begin{frame}
\frametitle{Componente Conductual}
\framesubtitle{Configurador Emocional}
\centering

\begin{exampleblock}{Actualización de Activación}
$A'(ag_i) = A_i + P_A$
\end{exampleblock}

\begin{alertblock}{Actualización de Satisfacción}
$ S'(ag_i) = S_i + P_S$
\end{alertblock}

\end{frame}

\begin{frame}
\frametitle{Componente Conductual}
\framesubtitle{Configurador Emocional}
\centering
Algoritmo del Configurador Emocional Para la Actualización del Estado Emocional del Agente
\begin{table}[!ht]
\tiny
\begin{tabular}{l}
\hline
\textbf{Entrada:} estímulo y BCC \\
\hline
Consultar en la BCC si el estímulo existe \\
\textbf{Si} existe conocimiento del estímulo \textbf{entonces:} \\
~~~~~PA  $\leftarrow$ obtener PA desde BCC \\
~~~~~PS  $\leftarrow$ obtener PS desde BCC \\
~~~~~Activación $\leftarrow$ Activación + PA \\
~~~~~Satisfacción $\leftarrow$ Satisfacción + PS \\
~~~~~Estado Emocional $\leftarrow$ Crear nuevo Estado Emocional a partir de los nuevos \\
~~~~~~~~~~~~~~~~~~~~~~~~~~~~~~~~~~~valores de Activación y Satisfacción \\
~~~~~Emoción $\leftarrow$ Consultar emoción en Modelo Afectivo a partir del nuevo Estado Emocional \\
\textbf{Fin si} \\\hline
\end{tabular}
\end{table}
\end{frame}

\begin{frame}
\frametitle{Componente Conductual}
\framesubtitle{Manejador de Comportamiento}
\centering
Algoritmo del Manejador de Comportamiento Para la Actualización de la Prioridad de Comportameinto
\begin{table}[!ht]
\tiny
\begin{tabular}{l}
\hline
\textbf{Entrada:} emoción provista por el configurador emocional \\
\hline
Consultar en la BCC la prioridad del comportamiento \\
\textbf{Si} existe comportamiento asociado a la emoción \textbf{entonces:} \\
~~~~~Comportamiento  $\leftarrow$ obtener comportamiento desde BCC \\
\textbf{Fin si} \\\hline
\end{tabular}
\end{table}
\end{frame}

\begin{frame}
\frametitle{Componente Conductual}
\framesubtitle{Manejador de Comportamiento}
\centering
\includegraphics[width=6cm]{ilustraciones/procesamiento-estimulo}
\end{frame}

\begin{frame}
\frametitle{Aspectos Nivel Colectivo}
\framesubtitle{Calculo de la Emoción Social}
\begin{exampleblock}{Emoción Social}
$ES(Ag) = \{EC(Ag), m(Ag), \sigma(Ag)\}$
\end{exampleblock}

Donde $Ag$ representa al grupo de agentes en estudio, $EC(Ag)$ se refiere a la
emoción central exhibida por el grupo de agentes, $m(Ag)$ es el estado emocional
más alejado de la $EC$, $\sigma(Ag)$ representa la dispersión emocional entorno
a la $EC$.
\end{frame}

\begin{frame}
\frametitle{Aspectos Nivel Colectivo}
\framesubtitle{Calculo de la Emoción Social}
\begin{exampleblock}{Emoción Central}
$EC(Ag) = (\bar A(Ag), \bar S(Ag))$
\end{exampleblock}

Donde $Ag$ representa al grupo de agentes en estudio, $\bar A$ es el promedio
de activación y $\bar S$ el promedio de satisfacción del grupo en estudio.
\end{frame}

\begin{frame}
\frametitle{Aspectos Nivel Colectivo}
\framesubtitle{Calculo de la Emoción Social}

\begin{exampleblock}{Promedio de la Activación}
$\bar A(Ag)=\frac{\sum_{i=1}^n A_i}{n}, \forall ag_i \in Ag$ \\
\end{exampleblock}

\begin{exampleblock}{Promedio de la Satisfacción}
$\bar S(Ag)=\frac{\sum_{i=1}^n S_i}{n}, \forall ag_i \in Ag$
\end{exampleblock}

Donde $Ag$ representa al grupo de agentes en estudio, $A_i$ es la activación y $S_i$ la satisfacción del agente $i$, para $1 \leq i \leq n$.

\end{frame}

\begin{frame}
\frametitle{Aspectos Nivel Colectivo}
\framesubtitle{Calculo de la Emoción Social}
\begin{exampleblock}{Distancia Máxima}
$ m(Ag) = (m_A(Ag), m_S(Ag))$
\end{exampleblock}

Donde $Ag$ representa al grupo de agentes en estudio, $m_A(Ag)$ es
la activación más alejada (máxima activación) y $m_S(Ag)$ es la satisfacción más
alejada (máxima satisfacción).
\end{frame}

\begin{frame}
\frametitle{Aspectos Nivel Colectivo}
\framesubtitle{Calculo de la Emoción Social}

\begin{exampleblock}{Distancia Máxima de la Activación}
$m_A(Ag) = max\left(\sqrt{(A_i - \bar A(Ag))^2}\right), \forall ag_i \in Ag$
\end{exampleblock}

\begin{exampleblock}{Distancia Máxima de la Satisfacción}
$m_S(Ag) = max\left(\sqrt{(S_i - \bar S(Ag))^2}\right), \forall ag_i \in Ag$
\end{exampleblock}

Donde $Ag$ es el grupo de agentes, $A_i$ es la activación y $S_i(Ag)$ la satisfacción del agente $ag_i$,
 $\bar A$ es el promedio de activación y $\bar S(Ag)$ el promedio de satisfacción del grupo en estudio.
\end{frame}

\begin{frame}
\frametitle{Aspectos Nivel Colectivo}
\framesubtitle{Calculo de la Emoción Social}
\begin{exampleblock}{Dispersión Emocional}
$ \sigma(Ag) = (\sigma_A(Ag), \sigma_S(Ag))$
\end{exampleblock}

Donde $\sigma_A(Ag)$ es la desviación estándar de la activación y
$\sigma_S(Ag)$ es la desviación estándar de la satisfacción del grupo
de agentes $Ag$.
\end{frame}

\begin{frame}
\frametitle{Aspectos Nivel Colectivo}
\framesubtitle{Calculo de la Emoción Social}

\begin{exampleblock}{Dispersión Emocional de la Activación}
$\sigma_A(Ag) = \sqrt{\frac{\sum_{i=1}^n(A_i - \bar A(Ag))^2}{n}},  \forall ag_i \in Ag$
\end{exampleblock}

\begin{exampleblock}{Dispersión Emocional de la Satisfacción}
$  \sigma_S(Ag) = \sqrt{\frac{\sum_{i=1}^n(S_i - \bar S(Ag))^2}{n}},  \forall ag_i \in Ag$
\end{exampleblock}

Donde $Ag$ es el grupo de agentes, $A_i$ es la activación y $S_i$ la satisfacción del agente $ag_i$,
para $1 \leq i \leq n$, $\bar A(Ag)$ es el promedio de activación y $\bar S(Ag)$
el promedio de satisfacción del grupo en estudio.
\end{frame}

\section{Casos de Estudio}
\begin{frame}
\frametitle{Casos de Estudio}

\end{frame}

\DoBlueBackgroundTitle{Demostración}

\section{Conclusión}
\begin{frame}
\frametitle{Aportes}

\end{frame}

\begin{frame}
\frametitle{Conclusión}

\end{frame}

\DoBlueBackgroundTitle{Preguntas}

\ThankYouFrame

\end{document}
